\documentclass[a4paper]{article}
\usepackage{amsmath,amsthm,amssymb}
\usepackage{hyperref}
\usepackage{fullpage}
\usepackage{enumitem}

\newcommand{\ii}{\textup{i}}									% \textup{i} := \sqrt{-1}
\newcommand{\tr}{\mathsf{T}}									% transpose T

\begin{document}

\textbf{Math 307 -- Problem Set 2 \hfill Name: Alexander Powell}

Printed-copy Due: 5pm on 09/18/2014

\begin{enumerate}[leftmargin=.5in,label=(\textbf{\arabic*})]
%1
\item Let $G$ be a group and $g \in G$. For all positive integers $n$, show that $(g^{-1})^n = (g^n)^{-1}$. 
\begin{proof}

We prove this by induction.  Base step: when $n=1$, it is clear that $(g^{-1})^{1} = g^{-1} = (g^{1})^{-1}$.  Now assume this is true for $n = k, k \in \mathbb{Z}, k > 1$.  Then let $n=k+1$, so we have that $(g^{-1})^{k+1} = (g^{-1})^{k} \cdot g^{-1} = (g^{k})^{-1} \cdot g^{-1} = (g^{k+1})^{-1}$.  

So, via the principle of mathematical induction, it is proven that for all positive integers $n$, $(g^{-1})^n = (g^n)^{-1}$. 

\end{proof}

%2
\item For $n \in \mathbb{N}$, $n>1$, let $\mathbb{Z}_n := \{0,1, \dots, n-1\}$ and $\mathbb{Z}_n^\times := \{1, \dots, n-1\}$.

\begin{enumerate}[leftmargin=*, label=(\textbf{\alph*})]
\item Show that $\left( \mathbb{Z}_n,+ \right)$, where $a+b:= (a+b)\bmod{n}$, is a group.

\begin{proof}

To prove something is a group, we must establish 4 things: closure, associativity, identity, and inverse relationships.  

We can revert back to the division algorithm, which states that modular addition is a binary operation, to prove closure.  

To prove associativity, let $a,b,c \in \mathbb{Z}_n$.  It can be shown that:
$((a+b)mod n + c)mod n = ((a+b)+c)mod n = (a+(b+c))mod n = (a+(b+c)mod n)mod n$.  

To prove identity, it is clear that for any element $m \in \mathbb{Z}_n, (m+0)mod n = (0+m)mod n = m$.  Therefore, an identity can be defined for every element in $\mathbb{Z}_n$.  

Finally, to prove the inverse, we can show that if $m \in \mathbb{Z}_n$, then it is the case that $(m+(n-m))mod n = n mod n = 0$

\end{proof}

\item \label{partb} For positive integers $a$ and $n$, show that $ax \bmod{n} = 1$ has a solution if and only if $\gcd(a,n)=1$.

\begin{proof}

First we prove that $ax \bmod{n} = 1$ has a solution if $\gcd(a,n)=1$.
Let $(ab)modn = 1, b \in \mathbb{Z}$.  Now we know there exists $p \in \mathbb{Z}$ such that $ab = np+1$ which can be rewritten as $ab-np=1$ and $gcd(a,n)=1$.  

Next we need to prove that if $\gcd(a,n)=1$, then $ax \bmod{n} = 1$ has a solution.  
Since $gcd(a,n)=1$, we know that there exist $x,q \in \mathbb{Z}$ such that $ax+nq=1$.  Also, $(ax+nq)mod n=1 mod n = 1 = ((ax)modn +(nq)modn)modn = ((as)modn + 0)modn = (as)modn$.  

\end{proof}

\item Use part \hyperref[partb]{(b)} to show that $(\mathbb{Z}_n^\times,\cdot)$, where $a\cdot b := (ab) \bmod{n}$, is a group if and only if $n$ is a prime.
\end{enumerate}

\begin{proof}

To prove closure: Let $a,b \in \mathbb{Z}_n$, then $(ab)modn \in \mathbb{Z}$.  Then we know that $ab \neq 0$.  If $(ab)modn = 0$, then there exists $p \in \mathbb{Z}$ such that $ab = np$.  

To prove associativity: Let $x,y,z \in \mathbb{Z}_n$, then 
$$((ab)modn \cdot c)modn = ((ab) \cdot c)modn = (a \cdot (bc))modn) = (a \cdot (bc)modn)modn$$.  

To prove identity: Let $i \in \mathbb{Z}_n$, then $(i \cdot 1)modn = (1 \cdot i)modn = i$.  

To prove inverse: From part b it is clear that $(ax)modn$ has an inverse for every $a \in \mathbb{Z}_n^\times$.  

\end{proof}

%3
\item Let $\mathbb{Q}(\sqrt{2}) := \{ a + \sqrt{2}b: a,b \in \mathbb{Q} \}$. Show that 

\begin{enumerate}[leftmargin=*, label=(\textbf{\alph*})]
\item $\mathbb{Q}(\sqrt{2}) \leq \mathbb{R}$. 

\begin{proof}

First it is necessary to prove that $\mathbb{Q}(\sqrt{2})$ is non-empty.  This is clearly the case because if you plug any rational numbers, $a$ and $b$, into the expression $a + \sqrt{2}b$, it will return a result.  Second, if we let $p,q \in \mathbb{Q}(\sqrt{2})$, then $p = a + \sqrt{2}b$ and $q = c + \sqrt{2}d$.  Then $(a + \sqrt{2}b)/(c + \sqrt{2}d) \in \mathbb{Q}(\sqrt{2}) \leq \mathbb{R}$.  

\end{proof}

\item $\mathbb{Q}(\sqrt{2})^\times \leq \mathbb{R}^\times$.

\begin{proof}

Again, it can easily be determined that $\mathbb{Q}(\sqrt{2})^\times$ is non-empty by simply entering in rational numbers for $a$ and $b$.  Also, we will again use $p,q \in \mathbb{Q}(\sqrt{2})^\times$, then $p = a + \sqrt{2}b$ and $q = c + \sqrt{2}d$.  The inverse of $q$ can be defined as $\frac{1}{(c + \sqrt{2}d)} = \frac{1}{(c + \sqrt{2}d)} \cdot \frac{(c - \sqrt{2}d)}{(c - \sqrt{2}d)} = \frac{(c - \sqrt{2}d)}{(c^2-2d^2)}$.  Then by taking the product of $x$ and $y^{-1}$, we get $$\frac{ac-\sqrt{2}ad}{c^2-2d^2}+\sqrt{2} \times (\frac{bc-\sqrt{2}db}{c^2-2d^2})$$ which is clearly an element of $\mathbb{Q}(\sqrt{2})^\times$.  

\end{proof}

\end{enumerate}

%4
\item Recall that the transpose of an $m \times n$ matrix $A = [a_{ij}]$, denoted by $A^\tr$,  is the $n \times m$ matrix whose entries are $[a_{ji}]$. Show that 
\[ O_n (\mathbb{R}):= \left\{ Q \in GL_n (\mathbb{R}) : Q^\tr Q = Q Q^\tr = I_n \right\} \leq GL_n (\mathbb{R}), \]
where $I_n$ denotes the $n \times n$ identity matrix.

\begin{proof}

First we must show that $O_n (\mathbb{R})$ is non-empty.  This is clearly the case because the identity matrix is an element of $O_n (\mathbb{R})$.  

Next, because $Q^\tr Q = Q Q^\tr = I$ we can determine that $Q$ is an orthogonal matrix.  Therefore, $Q^\tr=Q^{-1}$, so if we let $A$ and $B$ be matrices $\in O_n (\mathbb{R})$, then $AB^{-1}=AB^\tr$.  Now, we have that $(AB^\tr)^\tr AB^\tr = B(A^\tr A)B^\tr = BB^\tr = I$.  

This is an element of $O_n (\mathbb{R})$, thus $O_n (\mathbb{R}) \leq GL_n (\mathbb{R})$.  

\end{proof}






\end{enumerate}

\end{document}