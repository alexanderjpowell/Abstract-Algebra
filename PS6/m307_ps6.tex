\documentclass[a4paper]{article}
\usepackage{amsmath,amsthm,amssymb}
\usepackage[linkcolor=Blue, urlcolor=Blue,colorlinks=true]{hyperref}
\usepackage[usenames,dvipsnames]{xcolor}							% Fancy colors
\usepackage{fullpage}
\usepackage{enumitem}

% Macros
\newcommand{\ii}{\textup{i}}									% \textup{i} := \sqrt{-1}
\newcommand{\tr}{\mathsf{T}}									% transpose T
\newcommand{\bb}[1]{\mathbb{#1}}								% Blackboard boldface for R, C, N, Q
\newcommand{\inv}[1]{{#1}^{-1}}								% Inverse

\begin{document}

\textbf{Math 307 -- PS6 \hfill Name: Alexander Powell}

Printed-copy due: 5pm on 11/24/2014 

\begin{enumerate}[leftmargin=*, label=(\textbf{\arabic*})]
% 1
\item Let $\phi$ be a homomorphism from $G$ to $\bar{G}$ and $\sigma$ be a homomorphism from $\bar{G}$ to $\tilde{G}$. 

\begin{enumerate}[leftmargin=*, label=(\textbf{\alph*})]
\item Prove that $\psi:=\sigma \circ \phi$ is a homomorphism from $G$ to $\tilde{G}$.

\begin{proof}
Take $p$ and $q$ $\in G$.  Then we have that $\psi(pq) = \sigma(\phi(pq)) = \sigma(\phi(p)\phi(q)) = \sigma(\phi(p)) \sigma(\phi(q)) = \psi(p) \psi(q)$.  Therefore, $\psi:=\sigma \circ \phi$ is a homomorphism from $G$ to $\tilde{G}$.  
\end{proof}

\item Prove that $\ker(\phi) \leq \ker(\psi)$.

\begin{proof}
To prove that $\ker(\phi) \leq \ker(\psi)$ we need to establish that $\ker(\phi) \subseteq \ker(\psi)$.  By taking some $p \in \ker(\phi)$ then we have that $\psi(p) = \sigma(\phi(p)) = \sigma(\bar{e}) = \tilde{e}$.  Since $p \in \ker(\psi)$ we can say that $\ker(\phi) \subseteq \ker(\psi)$ and thus $\ker(\phi) \leq \ker(\psi)$ is established.  
\end{proof}

\end{enumerate} 

% 2 
\item Show that a homomorphism defined on a cyclic group is completely determined by its action on a generator of the group.

\begin{proof}
Take some homomorphism $h:G \rightarrow \bar{G}$.  If $\langle a \rangle = G$, then $h(a^{k}) = h(a)^{k}$, and since every element of $G$ is in the form $a^{k}$ their images are all determined by the image of $a$ and the product in the image group.  
\end{proof}

% 3
\item Let $\phi: \bb{R} \rightarrow SL_2 (\bb{R})$ be defined by
\begin{align*}
\phi(x) = 
\left[
\begin{array}{rr}
\cos x & \sin x \\
-\sin x & \cos x
\end{array}
\right].
\end{align*}

\begin{enumerate}[leftmargin=*, label=(\textbf{\alph*})]
\item Prove that $\phi$ is a homomorphism.

\begin{proof}
For any $x, y \in R$ then we can define 
$$\phi(x+y) = 
\begin{bmatrix}
\cos (x+y) & \sin (x+y) \\
-\sin (x+y) & \cos (x+y)
\end{bmatrix}$$
Then using basic trig properties we can rewrite the matrix as 
$$\begin{bmatrix}
\cos x \cos y - \sin x \sin y & \cos x \sin y + \sin x \cos y 	\\
-\sin x\cos y - \cos x\sin y & - \sin x\sin y + \cos x\cos y 
\end{bmatrix}
=
\begin{bmatrix}
\cos x & \sin x 	\\
-\sin x & \cos x
\end{bmatrix}
\times
\begin{bmatrix}
\cos y & \sin y 	\\
-\sin y & \cos y
\end{bmatrix}
$$
So we have that $\phi(x + y) = \phi(x) \phi(y)$, therefore $\phi$ is a homomorphism.  
\end{proof}

\item Prove that $\ker(\phi) = \langle 2 \pi \rangle$.

\begin{proof}
Any element $p \in \ker(\phi)$ if and only if 
$$
\begin{bmatrix}
\cos p & \sin p 	\\
-\sin p & \cos p
\end{bmatrix}
$$
is equal to the $2 \times 2$ identity matrix.  Since $\cos p = 1$ and $\sin p = 0$ only holds true if $p = 2 \pi k$, $k \in \bb{Z}$.  Therefore, it is established that $\ker(\phi) = \langle 2 \pi \rangle$.  
\end{proof}

\end{enumerate}

% 4
\item Let $\phi: \bb{C}^\times \longrightarrow \bb{C}^\times$ be defined by $\phi(z) = z^n$.

\begin{enumerate}[leftmargin=*, label=(\textbf{\alph*})]
\item Prove that $\phi$ is a homomorphism.

\begin{proof}
Take some $z$ and $y \in \bb{C}^\times$.  Then we can write $z = a(\cos \theta + \ii \sin \theta)$ and $y = b(\cos \psi + \ii \sin \psi)$ and $a,b > 0$ with $\theta, \psi \in [0,2\pi)$.  Via DeMoivre's theorem we have that 
$$\phi(zy)=(zy)^{n}=[a(\cos\theta + \ii \sin\theta)b(\cos\psi + \ii \sin\psi)]^n=a^n b^n [\cos(\theta + \psi) + \ii \sin(\theta + \psi)]^n$$
$$=a^n b^n [\cos(n\theta + n\psi) + \ii \sin(n\theta + n\psi)]$$
$$=a^n[\cos(n\theta) + \ii \sin(n\theta)] b^n [\cos(n\psi) + \ii \sin(n\psi)] = z^n y^n = \phi(z) \phi(y)$$
Therefore, it is established that $\phi$ is a homomorphism.  
\end{proof}


\item Prove that $\ker(\phi) = \Omega_n := \{ \exp(2 k \pi i / n) : k =0,1, \dots, n- 1\}$. 

\begin{proof}
Again, take $z = a(\cos \theta + \ii \sin \theta)$ and $y = b(\cos \psi + \ii \sin \psi)$ where both $x,y \in \bb{C}$.  Then the only case when $z=y$ is when $a=b$ and $\theta=2\pi k, k \in \bb{Z}$.  So, $z = a(\cos\theta + \ii \sin\theta) \in \ker(\phi)$ iff $z^n = 1$ so we can write $r^n (\cos(n\theta) + \ii \sin(n\theta)) = (\cos(2\pi k) + \ii \sin(2\pi k)), k \in \bb{Z}$.  This is only the case when $a^{n} = 1$ and $n\theta = 2\pi k$.  So, $z \in \ker(\phi)$ only when $z = \cos(2 \pi k/ n) + \ii \sin(2\pi k/n) = (\cos(2 \pi/n) + \ii \sin(2 \pi/n))^k, k \in \bb{Z}$, however $z = (\cos(2 \pi/n) + \ii \sin(2 \pi/n))^{k \bmod n}$.  So we have that $\ker(\phi) = \Omega_n := \{ \exp(2 k \pi i / n) : k =0,1, \dots, n- 1\}$.  
\end{proof}

\item Prove that $\bb{C}^\times/\Omega_n \cong \bb{C}^\times$.

\begin{proof}
If $z = a(\cos\theta + \ii\sin\theta)$ and $y = a^{1/n}(\cos(n\theta) + \ii\sin(n\theta))$ and both $z,y \in \bb{C}^\times$, then $\phi(w) = z$.  Therefore $\phi(\bb{C}^\times) = \bb{C}^\times$ and $\bb{C}^\times/\Omega_n \cong \bb{C}^\times$.  
\end{proof}

\end{enumerate}

\end{enumerate}

\end{document}